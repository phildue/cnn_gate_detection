\chapter{Speed Optimization}
\label{sec:tradeoff}

\begin{center}
	\textbf{What are the trade-off's between detection accuracy and inference time when such a detection model is applied on a micro-air vehicle?}
\end{center}

A major drawback of \acp{CNN}-based object detectors is their vast resource requirements. In recent years a lot of research has been conducted to reduce the inference time of \acp{CNN}. The publications address different levels for optimization:
\begin{enumerate}
	\item \textbf{Conceptual Level}
	\item \textbf{Architectural Level}
	\item \textbf{Operational Level}
\end{enumerate}

\section{Conceptual Level}
 On a conceptual level effort has been carried out to incorporate more steps of the object detection pipeline into one model while sharing computational load. This has led from Overfeat\todoref{overfeat} which ran a \acp{CNN} in sliding window manner across the image over Faster-RCNN\todoref{faster rcnn} where the features are extracted only once and reused to Yolo and SSD that combine the whole pipeline in one model.
 Using Time domain:
 \cite{Chen2018}
 
\section{Architectural Level}

Architectural Level/Number of Computations:
\cite{Howard2017}, \cite{Zhang2017a}, \cite{Ghosh2017}, 
\todo{Knowledge Distillation}

\section{Operational Level}
Operational Level - Quantization:
\cite{TripathiSanDiego}, 



\todo{We choose one/two of the above because trying everything is a bit too much. So which one and why?}

\section{Experiments}

\todo{Evaluate effects on performance and accuracy}

\section{Conclusion}