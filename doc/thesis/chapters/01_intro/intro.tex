\chapter{Introduction}
\label{sec:intro}

Recent developments in embedded systems and artificial intelligence enable micro air vehicles (MAVs) to act more and more autonomously. In the near future autonomous air vehicles will be used in many applications such as automated warehouses, package delivery or even search and rescue scenarios.

The research on algorithms that allow these kind of technologies gave rise to autonomous drone races, where researchers have the opportunity to evaluate their algorithms in a competitive environment.

The currently most challenging autonomous drone race is held yearly at the IROS conference. The race track consists of several coloured metal gates which need to be passed one after another. The participating drones have to fly fully autonomously with all processing happening on board.

An important part to fulfil this task is an accurate estimation of the MAV's state within its environment. Sensors like inertial measurement units and sonars can deliver accurate data of the drones altitude. However, to determine the MAV's global position these sensors are usually too subjective to noise. As the race happens indoor in a GPS-denied environment the most accurate sensor to determine the global position is a camera. Hence, high-performing vision algorithms are crucial in order to complete the race court as fast as possible. 

Within the field of computer vision big advances could be achieved using supervised machine learning. Here mathematical models are trained using labelled examples of the object of interest. These learned algorithms tend to outperform most human crafted methods by a far extent. However, they are heavily dependent on examples used during training.

This thesis investigates a learning based vision method in order to complete the race track of the IROS 2018 autonomous drone race. The goal is to get an accurate estimation of the drones position with respect to the racing gates. The estimated position is then used by the control system to generate the optimal trajectory.

Racing MAV's have only limited computational resources available and need to be fast by definition. Hence, the developed algorithm needs not only to be accurate but also computationally cheap.

From the task at hand an interesting pattern recognition problem arises. In contrast to most other objects the object of interest consist only of a small frame on the border while large parts of it are occupied by background. The detection of these objects enables the application of tailored algorithms and is not yet investigated by the research community in detail.

Due to the limited attention in research, the amount of annotated datasets containing racing gates is limited. To this end no annotated dataset is publicly available. As the object is largely occupied by background, a dataset should contain different shaped and colored gates in a large variety of environments. Otherwise there is always the danger for learning based methods to overfit to the background. This makes the creation of such a dataset quite expensive.

The challenges tackled by this thesis can be summarized in:
\begin{enumerate}
	\item The object of interest consists only of several thin metal poles. Thus not many distinct features of the object can be identified.
	\item No large annotated datasets are available. 
	\item Only the arrangement of gates is known a priori. Environmental factors like light conditions and spectators can influence the performance of the vision system a lot. The developed vision system should be robust against these kind of influences.
	\item The hardware resources on the MAV are quite limited and within the race the drone needs to be fast by definition. Hence, any vision system should maximally exploit the available resources.
\end{enumerate} 

%TODO describe some related work

.. We formulate the research question as follows:

\textbf{How can we efficiently detect featureless objects when no annotated datasets are available and the computational resources are limited?}


The remaining parts of this thesis are structured as follows: \autoref{sec:background} outlines background and related work, \autoref{sec:method} describes the proposed method, \autoref{sec:evaluation} contains the  experiments that have been carried out, \autoref{sec:concl} discusses the results and formulates a conclusion.
