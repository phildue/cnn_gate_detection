%%%%%%%%%%%%%%%%%%%%%%%%%%%%%%%%%%%%%%%%%
% University/School Laboratory Report
% LaTeX Template
% Version 3.1 (25/3/14)
%
% This template has been downloaded from:
% http://www.LaTeXTemplates.com
%
% Original author:
% Linux and Unix Users Group at Virginia Tech Wiki 
% (https://vtluug.org/wiki/Example_LaTeX_chem_lab_report)
%
% License:
% CC BY-NC-SA 3.0 (http://creativecommons.org/licenses/by-nc-sa/3.0/)
%
%%%%%%%%%%%%%%%%%%%%%%%%%%%%%%%%%%%%%%%%%

%----------------------------------------------------------------------------------------
%	PACKAGES AND DOCUMENT CONFIGURATIONS
%----------------------------------------------------------------------------------------

\documentclass{article}

\usepackage[version=3]{mhchem} % Package for chemical equation typesetting
%\usepackage{siunitx} % Provides the \SI{}{} and \si{} command for typesetting SI units
\usepackage{graphicx} % Required for the inclusion of images
\usepackage{natbib} % Required to change bibliography style to APA
\usepackage{amsmath} % Required for some math elements 
\usepackage{hyperref}
\usepackage[a4paper,margin=0.5in]{geometry}
\setlength\parindent{0pt} % Removes all indentation from paragraphs

\renewcommand{\labelenumi}{\alph{enumi}.} % Make numbering in the enumerate environment by letter rather than number (e.g. section 6)

%\usepackage{times} % Uncomment to use the Times New Roman font

%----------------------------------------------------------------------------------------
%	DOCUMENT INFORMATION
%----------------------------------------------------------------------------------------

\title{Gate Detection} % Title

\author{Philipp \textsc{Duernay}} % Author name

\date{\today} % Date for the report

\begin{document}
\maketitle
% If you wish to include an abstract, uncomment the lines below
% \begin{abstract}
% Abstract text
% \end{abstract}

%----------------------------------------------------------------------------------------
%	SECTION 1
%----------------------------------------------------------------------------------------

\section{Recap}
In the last meeting from 28.03.2018 several next steps were defined:
\begin{enumerate}
	\item \textbf{Generate data with Airsim}
	\item \textbf{Compare performance when gate contains an image}
	\item \textbf{Use ssd from tensorflow-object-detection api}
	\item \textbf{Rethink approach, How to define an object?}
\end{enumerate}

\section{Data Generation}
Getting ground truth bounding boxes out of the simulator took a bit more time than expected since this is not provided by default and how the graphic engine renders the scene is not very well documented. However, in the end ground truth labels could be obtained. Two scenes were created for training and testing the models. Examples are shown in \autoref{fig:scenes}
\begin{figure}
	\begin{minipage}{0.5\textwidth}
		content...
	\end{minipage}
	\begin{minipage}{0.5\textwidth}
		content...
	\end{minipage}
	\label{fig:scenes}
\end{figure} 

\section{Evaluation}
We assume the reason for the low performance so far is the absence of higher order features. Deep object detectors like Yolo rely on the assumption that the lower layers with a small receptive field detect basic shapes like edges and corners, while higher layers combine those shapes in a deeper and deeper semantic representation. However, the object we want to detect basically consists of 4 corners and 4 lines in orange colour, whereas the biggest part of the bounding box that surrounds the object is occupied by background. 

In order to proof this hypothesis we do an experiment. We place an image with a more complex object inside the area that is occupied by background. Ideally the image itself is not too easy to detect such that the model has to learn a deeper representation. An example of the chosen image can be seen in \autoref{fig:cats}.

%

\section{Conclusion}



%----------------------------------------------------------------------------------------


%----------------------------------------------------------------------------------------
%	BIBLIOGRAPHY
%----------------------------------------------------------------------------------------

\bibliographystyle{abbrv}

\bibliography{literature}

%----------------------------------------------------------------------------------------


\end{document}\grid
