\chapter*{Summary}
\addcontentsline{toc}{chapter}{Summary}
\setheader{Summary}

Object detection has been an intensively studied area within computer vision. The determination of what kind of objects we see where in an image is an everyday human task, that machines still have a hard time at solving.

In recent years deep learning based object detectors have been pushing state of the art results tremendously after a long time of almost no improvements. The reason anticipated to be responsible for the strong performance boost is partly the flexibility of the model but mostly the fact that the whole detection pipeline can be integrated into one model and trained end-to-end.

The hereby used convolutional neural networks are designed in such a way that lower order layers look at small regions of the image and determine the presence of simple shapes like edges corners and blobs. Higher order layers look at the activations of the lower order stages and combine the responses to more complex shapes reaching from object parts to whole objects.

Hence, the power of deep learning based object detection arises from the complexity of real-world objects that can be efficiently represented by convolutional neural networks using a tremendous amount of parameters. The advances in new computer architectures and GPU technology and the availability of huge datasets due to the rise of the internet allowed the efficient training of these huge models.

However, there are applications where each of the above is only available to a certain extent. For example mobile phones, robots and micro-air vehicles need to exploit their resources efficiently. Firstly because saving energy allows for a longer lifetime, secondly because these systems usually have real-time constraints that can't be met when the used algorithms are too slow.

Furthermore, there are objects that don't consists of complex shapes or patterns but are built from very simple geometry. Examples for these object which we refer to as "wire frame objects" are antennas, wires, fences, cages and the like. 

This thesis studies object detection in an extrem example of the aforementioned case: Gate Detection at the IROS 2018 Autonomous Drone Race. Within this race court several coloured metal gates are placed and need to be passed one after another as fast as possible. Hence, the more lightweight the algorithm can be executed, the less computational resources are required the lighter the MAV can be build and the more aggressive manoeuvres can be flown. On the other hand if the method is too inaccurate, the drone misses a gate and is likely to loose the race or even crash.

From the application on MAV's further issues arise: MAV-specific camera properties like lense distortion or blur due to motion leads to images with drone specific noise. ..


