\documentclass{article}
\usepackage{lscape}
\usepackage{gensymb}
\usepackage{textcomp}
\usepackage{multicol}
\usepackage[a4paper,margin=0.5in]{geometry}
\begin{document}
	\title{Literature Review}
	\author{Philipp Duernay}
	\maketitle
	\begin{multicols}{2}
		
	\section{Summaries}
	\subsection{Object Detection}
		\subsubsection{Scalable Object Detection using Deep Neural Networks\cite{Erhan}}
		\begin{itemize}
			\item[-] Generates number of bounding boxes as object candidates (class agnostic) and confidences for each box
			\item[-] For each Bounding Box a classifier is run e.g. DNN
			\item[-] Training: If the number of boxes k is larger than the number of objects b, only b boxes are matched while the confidence of the others is minimized
			\item[-] Assignment problem $$F_{match}(x,l) = \frac{1}{2}\sum_{i,j}x_{ij}||l_i - g_j||^2_2$$ where $x_ij$ is one if the ith prediction is assigned to the jth ground truth object
			\item[-] Confidence: 
			$$F_{conf}(x,c) = - \sum_{i,j}x_{ij}*\log(c_i)-\sum_{i}(1-\sum_{j}x_{ij})\log{1-c_j}$$
			\item[-] Speed up training by clustering (kmeans) of ground truth and using it as prior (prior matching)
			\item[-] Can be defined to output boxes only for a particular class by training the bounding boxes on that class
			\item[-] Number of parameters grows linearly with number of classes
			\item[-] Authors argue two step process (region proposal + classification) is better
			\item[-] Architecture based on AlexNet
			\item[-] Predicted boxes are merged using non-maxima surpression
			\item[-] One shot(50\%), +2scales (75\%)
			\item[-] OverFeat/ Selective Search are faster but much more expensive
		\end{itemize}
		\subsection{Yolo}
		TODO
		\subsection{Yolo v2}
		TODO
			 \subsubsection{Single Shot Multibox Detector \cite{Liu}}
			 \begin{itemize}
			 	\item[-] Evaluates feature maps with different scales for all (a few) boxes in the image
			 	\item[-] Scale of feature map decreases each layer (feature map with different scales is a key difference to yolo and overfeat)
			 	\item[-] Based on VGG-16
			 	\item[-] Uses convolutional layers for classification instead of fully connected layers (yolo)
			 	\item[-] Similar to other box predictors, the ground truth box has to be chosen for training. Here this is done with the jaccard overlap (>0.5). Boxes can overlap.
			 	\item[-] LOSS FCN
			 	\item[-] Uses lower level feature maps in later state for prediction
			 	\item[-] Data augmentation significantly increases performance (~9\%)
			 	\item[-] More default boxes is better
			 	\item[-] Uses atrous algorithm to cover up holes when changing top layer
			 	\item[-] Replaces pooling with feature map at different scales
			 	\item[-] Non-maxima surpression at the end to get rid of the big amount of boxes (1.7ms)
			 	\item[-] Most time spent in base network and nms
			 	\item[-] Default boxes can have different aspect ratios
			 \end{itemize}
	\subsection{(Re-) Localization}
	\subsubsection{PoseNet: A Convolutional Network for Real-Time 6-DOF Camera Relocalization \cite{Kendall}}
	\begin{itemize}
			\item[-] Relocalizes, is trained on images from the scenes where it is applied	\item[-] Accuracy 2m and 3$\degree$  in 50 km$^2$ outdoors, 0.5m and 5$\degree$ indoors, 5ms per frame
		\item[-] ConvNet 23 layers, Image resolution 224x224
		\item[-] transfer learning from recognition/classification datasets (ConvNet is trained on classification tasks)
		\item[-] based on GoogleNet, affine regressors instead of softmax\item[-] automatic training data generation (structure from motion)
		\item[-] learns p from arbitrary global reference frame
		\item[-] $loss(I) = || \hat{x}-x||_2 + \beta*||\hat{q}-\frac{q}{||q||}||_2$
		\item[-] separating position/orientation led to drop in performance
		\item[-] PoseNet evaluation at single center crop + Dense PoseNet 128 uniformly spaced crops (time increase 95ms, only slight accuracy increase)
		\item[-] Training data generated using structure from motion (Cambridge Scene) and 7 Scenes (Microsoft) for indoor

	\end{itemize}
	 \subsubsection{A Deep Learning Based 6 Degree-of-Freedom
	 	Localization Method for Endoscopic Capsule Robots \cite{Turan2017}}
	 	\begin{itemize}
	 		\item[-] not published yet?
	 		\item[-] Uses 6-DOF camera pose directly
	 		\item[-] based on GoogleNet (9 Inception modules) trained on ImageNet
	 		\item[-] $loss(I) = ||\hat{x}-x||_2 + ||\hat{q}-q||_2$
	 		\item[-] Dataset of 10 000 frames taken from LM103 - EDG (EsophagoGastroDuodenoscopy) Simulator
	 		\item[-] 0.18cm RMSE on a trajectory of 18cm
	 		\item[-] Although 3 different cameras are used and the frames are separated for training and testing, its still the same "stomach". With 10 000 frames on a trajectory of 18 cm, won't the system just recognize the position?
	 		\item[-] Ground truth determined by seperate cameras 
	 	\end{itemize}
	 	\subsection{Object Pose Estimation}
	 	\subsubsection{3D generic object categorization localization and pose estimation \cite{Savarese}}
	 	\begin{itemize}
	 		\item[-] Other approaches use different class for different poses
	 		\item[-] Object model is separated in different parts of the object based on different view points (front view)
	 		\item[-] Different parts are connected when another part is visible from the front view via affine transformation
	 		\item[-] Generally such models can't handle inter class variations very good or increase in complexity as number of parts is increased. In this paper this is apparently not the case
	 	\end{itemize}
	 	\subsubsection{Uncertainty-Driven 6D Pose Estimation of Objects and Scenes from a Single RGB Image
	 		Eric \cite{Brachmann}}
	 	\begin{itemize}
	 	\item[-] Intermediate representation are object coordinates, continious part labeling that are jointly regressed for every pixel in the image
	 	\item[-] Based on auto context (Classifiers with several stages)
	 	\item[-] (1) (Auto context) Random forest with L1 regularization predicts labels and object coordinates for every pixel (2) Ransac predicts poses from 2d-3d correspondences guided by uncertainty labels
	 	(3) Refinement
	 	\item[-] Random forest predicts (probability to belong to object + 3d coordinate|given belonging to object)
	 	\item[-] Stacked Forests (Auto context) refine output on previous smoothed output (Geodesic Forest). The smoothing is done to enforce coupling of neighbors 
	 	\item[-] RANSAC formulates hypothesis by drawing 4 correspondences and solving PnP
	 	\item[-] Outperforms PoseNet in indoor localization
	 	\item[-] 6D within 5cm and 5 degree only 40 \% (With RGB-D 82.5\%), on other set 50 \% with unknown scene average median error 8.5cm 3.3°
	 	\item[-] Biggest translational error in z direction
	 	\item[-] Multi object detection/pose estimation in 1-4 seconds, not optimized, most time spend in searching for object hypothesis
	 	\end{itemize}
	 	\subsubsection{A Comparative Analysis and Study of Multiview CNN Models for Joint Object Categorization and Pose Estimation\cite{Elhoseiny}}
	 	\begin{itemize}
	 		\item[-] While detection needs pose invariant features, pose estimation needs the pose
		 	\item[-] Single instance 3d model
		 	\item[-] Discrete pose approaches (pose as classification)
		 	\item[-] Trains pose regressor and classifier on output of different levels to measure quality of features
		 	\item[-] Later layers "forget" about pose, paper suggests early branching
	 	\end{itemize}
	\subsection{Other}
		 \subsubsection{Deformable Convolutional Networks \cite{Dai}}
		 \begin{itemize}
		 	\item[-] Addresses problem of modeling geometric transformations
		 	\item[-] Introduces \textit{Deformable Convolution} which adds 2D offsets to the regular sampling grid. The offsets are learned from the data.
		 	\item[-] Introduces \textit{Deformable RoI pooling} which adds offsets to bins of pooling layers. The offsets are also learned from the data.
		 	\item[-] Further alternatives to have more variable feature maps: Spatial Transformer Networks, Active Convolution, Effective Receptive Field, Atrous Convolution, DeepID-Net, Spatial manipulation in RoI pooling (handcrafted), DPM (handcrafted)
		 	\item[-] Light-weight version of STN, easier to train and to integrate
		 	\item[-] Receptive fields seem to scale with the size of objects
		 	\item[-] Model complexity is increased by only 1-2%
		 	\end{itemize}
		 	
\end{multicols}	
\begin{landscape}

\begin{table}[]
	
	\caption{Object Detection}
	\label{my-label}
	\begin{tabular}{|p{3cm}|p{3cm}|p{3cm}|p{3cm}|p{3cm}|p{3cm}|p{3cm}|p{3cm}|}
		\hline
		& \multicolumn{3}{l|}{Traditional} & \multicolumn{4}{l|}{Deep}   \\ \hline
		& Viola\&Jones    				   & HoG    & DPM   		   & R-CNN    & YOLO         & SSD & OverFeat \\ \hline
		Feature Detector & Haar					   & HoG    & Multiple Hogs and virtual springs   & Learned by CNN     &  Learned by CNN            & & \\ \hline
		Detection & \multicolumn{3}{l|}{Sliding Window, high filter responses indicate there is an object} & NN in sliding window detects regions for possible objects, For each proposed region a classification is run & Image is split in Grid each Grid spawns Bounding boxes and gives class probabilities & & \\
		\hline
		Accuracy (voc) &  & & & 73.2 mAP & 63.4 mAP & 74.3 mAP & \\ 
		\hline
		Speed & & & & 7 FPS (Faster-RCNN) & 45 FPS & 59 FPS & \\
		\hline
		Strengths & & & & & &  &\\
		\hline
		Weaknesses & & & & & &  &\\
		\hline
		\end{tabular}
		
		\end{table}
		\end{landscape}	
\bibliography{literature.bib}
\bibliographystyle{acm}

\end{document}
