\chapter{Conclusion \& Future Work}
\label{sec:conclusion}
This work investigated the detection of \acp{EWFO} on \ac{MAV} using a \ac{CNN}. In this section a final discussion is given and a conclusion is derived. Furthermore, the research questions are answered and possible future work is discussed.

The research was motivated by the promising results of Deep Learning based Object Detectors and several drawbacks of manually crafted algorithms for the detection of racing gates in \ac{MAV} races. As the manually tuned features prove to be sensitive to light changes as well as to object occlusion, the aim was to investigate a more robust method. 

As no real training data were available, images and corresponding object labels were created with a simulator in order to train the \acp{CNN} based Object Detector \textit{YoloV3}. We hypothesized that due to the simple shape of \acp{EWFO} a small network should be able to learn the detection task. This led to the two research questions of this work which are now answered based on the conducted experiments.

\begin{enumerate}
	\item[\textbf{RQ1}]How can data be generated to train an object detector for \ac{EWFO} detection on a \acp{MAV}?
	
	It could be seen how the detector is sensitive to overfit to environmental conditions present in the training set. When testing a detector in a different simulated environment than the training set, the performance deteriorates between 30\% and 70\%. It was further investigated how to make the detector more invariant against such environmental changes. The results show how the variance in background is less important than the creation of realistic light conditions in the training set.
	
	When a wide range of view angles is introduced in training and test sets, the performance drops particularly for larger objects. It seems the detector has difficulties learning the detection from many angles. Better performance can be achieved by reducing the number of view angles in the training set. As this raised the question of how to create realistic view points, we proposed to simulate a flight through a race court. This way the created samples resemble the real world better. Even on unseen race courts, the detector achieves a precision of 70\% compared to the 20\% achieved by the network trained without simulating a flight. 
	
	Simulating flights as well as randomly placing the camera lack of control about the actual samples present in the training set. It is hard to think about all the view points that are required to train the network for a \acp{MAV} race. However, more control about the view points could give more insights about how the detector performs with different view angles in training and test set. Hence, future work could extend the data generation tool with more control about the view points and to conduct further experiments in that respect.
		
	In order to transfer the detector to the real world, it was found that image augmentation is crucial. Particularly modelling distortion improved the results on the real data. Yet there remains a gap between the results obtained in simulation and on real data. It cannot fully be resolved whether this is because of the complexity of the real data, or whether there are certain properties missing in the data generation process. Future work could address this issue by including real data in the training process. If this improves the results significantly, the problem is likely because of a reality gap. Otherwise, there might be a more fundamental problem in the chosen detector/complexity of the test set.
	
	In summary, we propose to fully synthesize environments when creating training data for the detection of \acp{EWFO} on \acp{MAV}. Furthermore, the precision can be improved by training the detector based on view angles it will see in the real world, possibly by simulating the flight behaviour. In order to transfer the detector to the real world, we recommend to use image augmentation. Particularly augmenting the images by modelling lens distortion improves the performance on the investigated dataset.

	\item[\textbf{RQ2}]How can \acp{EWFO} be detected using a \ac{CNN} on a \acp{MAV}?
	
	The \textit{YoloV3} can be adapted to the detection of \acp{EWFO}. The experiments showed how the detector can be confused by structures that are present within the empty part of the object. This was resolved by providing samples from different backgrounds and light conditions. We hypothesized that many backgrounds are required to achieve background invariance. However, the experiments showed that a small number of different environments is already enough to make the detector more robust against such confusion.
		
	A general drop of recall was observed for closer objects. It can be assumed that this is because it is more likely that object parts are out of view when the object comes closer. For the detection of racing gates on \acp{MAV} this result can be taken into account in later stages of the control loop. For example by using a dedicated detector for closer gates and combining the information. For the general detection of \acp{EWFO}, this result is interesting. Despite being clearly visible for a human, the detector struggles in some examples. Further experiments could include investigating how the performance changes when the detector is trained without any context such as the object pole.
	
	As \acp{EWFO} consist of relatively simple features, we hypothesized that a small network should be able to learn the task. The results showed that a shallow network of 9 layers performs equally well than a network with 15 layers. However, further reducing depth gradually reduces performance from 51\% to 35\%. When reducing the width of a network with 9 layers it can be seen how the performance slowly deteriorates from 51\% at its original size to 38\% to a fraction $\frac{1}{16}$ of its original size. 
	Hence, we conclude that a minimum of 9 layers is required if detection performance is the most important metric.
	
	For the detection of \acp{EWFO} on \acp{MAV} computational resources are more important. Our results show that a network with only one filter in the first layer can still achieve 38\% average precision. Hence, by reducing the network size to 0.4\% of its original size the performance drops only by 13\% average precision.
	
	The reduction of network size was taken out by retraining the network with a thinner architecture. An alternative way of reducing the network size is to apply \textit{Knowledge Distillation} (\Cref{sec:background}). This method showed promising results on deeper network. In future work it could similarly be applied for the detection of \acp{EWFO}.
	
	The trained detector was deployed on an example \ac{MAV}. The results show how by reducing the resolution the inference time can be increased by 45 ms/frame. However, the costs in terms of average precision are large as only 18\$ average precision can be achieved. Instead we propose to remove pooling layers and to use convolutions with larger strides. This parameter allows to trade-off between average precision and inference speed. An overview of the trade-off is given in \Cref{fig:ap_speed_tradeoff}.
	
	An alternative way to increase the network speed is \textit{weight quantization} (\Cref{sec:background}). As the target system of this work supports floating point multiplications, we did not further consider it. However, with an adequate low level implementation this could still lead to further speed up and should be investigated in future work.
	
	In a nutshell, it could be seen that a small network is able to learn the task. The inference time of the detector can be further increased without loosing too much performance. Finally, replacing pooling layers by convolutions with larger strides allows to trade-off between detection performance and inference speed.
		
\end{enumerate}

Based on the experiments we can give recommendations about the generation of data for the detection of \acp{EWFO}. Furthermore, we have insights of the limitations of the detector for example a drop of recall for larger object sizes. These insights can be taken into account when using the detector in a control loop. Finally, a detector could be developed and compared against the baseline. The experiments showed an improvement of performance compared to \textit{SnakeGate} of up to 16\% average precision, leading to a total of 32 \% $ap_{60}$. This improvement is mainly obtained in cases of difficult light conditions or occlusion. Hence, it can be concluded that indeed the \acp{CNN} can work better in such situations.

A network of similar complexity achieves 41\% $ap_{60}$ on a simulated data set which contains more objects and more difficult view angles. Therefore it can be seen that the potential of the learning based detector is much higher. Yet transferring the detector to the real world proves to be difficult, despite the relatively simple features of \acp{EWFO}. In some cases the object is clearly visible but not detected by the \acp{CNN}. These results show the general drawback of Deep Learning based approaches. It is not transparent why the objects are not detected and what exactly was learned by the network.

A frequent argument for Deep Learning is that it does not require cumbersome Feature Engineering. In this work, this step was technically replaced by Data Engineering and yet the remaining reality gap is high and some results are hard to understand. We have to ask ourselves if the lack of transparency, the amount of required data as well as the computational requirements are really worth the results gained in detection performance. 

Nevertheless, this work serves as a baseline for future work. The initial experiments show how a small amount of environments is already enough to make the detector relatively invariant against background. These results should apply similarly to the real world. Hence, it is not too much work to create a real world training set, and the data could be augmented with the data created in this work.

Furthermore, the experiments show trade-offs in speed and detection performance. The results can be used to design a detector for \acp{EWFO} based on available hardware and project requirements.




