\chapter*{Summary}
\addcontentsline{toc}{chapter}{Summary}
\setheader{Summary}

\ac{MAV} are an emerging technology that supports a wide range of applications. Thereby the robust estimation of an \ac{MAV}'s state within its environment is crucial to ensure safe operation. In  indoor scenarios cameras are one of the predominant choices for state estimation sensors. This requires Computer Vision algorithms to interpret the obtained high dimensional signal. An application that allows the competitive evaluation of control and state estimation algorithms is \ac{MAV} Racing such as the \ac{IROS}2018 Autonomous Drone Race. Thereby a race court consisting of several race gates has to be followed. For a fast flight during such a race court the detection of the racing gates with a camera can be used in a high level control loop. As these object consist only of small structures that are spread across large parts of the image, this gives rise to a challenging Object Detection problem.

In recent years \acp{CNN} showed promising results on various vision tasks. However, due to their computational complexity the deployment on mobile devices remains a challenge. Furthermore, \acp{CNN} typically require a vast amount of training data. This work defines the class of \acp{EWFO} and studies their detection on \acp{MAV} with \ac{Yolo}V3. For the training, data is generated with a graphical engine. 

The experiments conducted in simulation show how \acp{EWFO} are harder to detect than filled objects as the detector can be confused to patterns present in the empty part. Particularly for larger objects the detection performance decreases. We give several recommendations how to generate data for the detection of \acp{EWFO} on \acp{MAV}. These include how to include variations in background as well as the camera placement. Finally, we study the incorporation of image augmentation techniques to transfer the detector to the real world. We can report that especially modelling lens distortion improves the performance on the real data. Nevertheless, a reality gap remains that can not fully be explained.

Furthermore, different architectures are studied for the detection of \acp{EWFO}. It can be seen how a relatively shallow network of 9 layers can be used for the detection of \acp{EWFO} on \acp{MAV}. A further reduction in weights leads to a gradual decrease in performance. 

Based on the gained insights the deployment of a detector on an example system is studied. A detection performance/speed trade-off is evaluated. The final detector achieves 32\% $ap_{60}$ at a frame rate of 12 Hz on a real world test set created during this work.


