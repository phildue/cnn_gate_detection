%%%%%%%%%%%%%%%%%%%%%%%%%%%%%%%%%%%%%%%%%%%%%%%%%%%%%%%%%%%%%%%%%%%%
%% I, the copyright holder of this work, release this work into the
%% public domain. This applies worldwide. In some countries this may
%% not be legally possible; if so: I grant anyone the right to use
%% this work for any purpose, without any conditions, unless such
%% conditions are required by law.
%%%%%%%%%%%%%%%%%%%%%%%%%%%%%%%%%%%%%%%%%%%%%%%%%%%%%%%%%%%%%%%%%%%%

\documentclass{beamer}
\usetheme[faculty=fi]{fibeamer}
\usepackage[utf8]{inputenc}
\usepackage[
  main=english, %% By using `czech` or `slovak` as the main locale
                %% instead of `english`, you can typeset the
                %% presentation in either Czech or Slovak,
                %% respectively.
  czech, slovak %% The additional keys allow foreign texts to be
]{babel}        %% typeset as follows:
%%
%%   \begin{otherlanguage}{czech}   ... \end{otherlanguage}
%%   \begin{otherlanguage}{slovak}  ... \end{otherlanguage}
%%
%% These macros specify information about the presentation
\title{Robust Gate Detection for Autonomous Drone Racing} %% that will be typeset on the
\subtitle{Mid-Term Presentation} %% title page.
\author{Philipp Dürnay\\ Supervisors:	 David M. J. Tax, Guido de Chroon}
%% These additional packages are used within the document:
\usepackage{ragged2e}  % `\justifying` text
\usepackage{booktabs}  % Tables
\usepackage{tabularx}
\usepackage{tikz}      % Diagrams
\usetikzlibrary{calc, shapes, backgrounds}
\usepackage{amsmath, amssymb}
\usepackage{url}       % `\url`s
\usepackage{listings}  % Code listings
\frenchspacing
\begin{document}
  \frame{\maketitle}

  %\AtBeginSection[2,3,4]{% Print an outline at the beginning of sections
  %  \begin{frame}<beamer>
  %    \frametitle{Outline for Section \thesection}
  %    \tableofcontents[currentsection]
  %  \end{frame}}

  \begin{darkframes}
  	\begin{frame}{Outline}
  		\tableofcontents
  	\end{frame}
	\section{Introduction}
	\subsection{Application}
    \begin{frame}{IROS2018}
    	\includegraphics[width=\textwidth]{fig/application}
    \end{frame}
    \begin{frame}{Current Method}
       	content...
       	show loop
    \end{frame}

	\subsection{Research Question}    
	\begin{frame}{Research Question}
	\textbf{How can we efficiently detect thin frame objects on a micro-air vehicle?}
	\end{frame}
    \section{Background}
\subsection{Object Detection}
	\begin{frame}{Object Detection}
	\includegraphics[width=\textwidth]{fig/ObjectDetection}
	\end{frame}
	\subsection{Convolutional Networks}
		\begin{frame}{Convolutional Networks}
	\includegraphics[width=\textwidth]{fig/cnn}
\end{frame}

    
    \section{Approach}
  \subsection{Method}
    \begin{frame}{Approach}
  show reduced network mention weights
    \end{frame}
    
    \subsection{Results}
    \begin{frame}{Results}
    	\includegraphics[width=\textwidth]{fig/pr}
    \end{frame}

    \begin{frame}{Results - Examples}
    	\includegraphics[width=\textwidth]{fig/examples}
	\end{frame}
    
    \section{Conclusion}
    \begin{frame}{Conclusion}

    \begin{itemize}
    	\item Deeper layers don't serve much but objects in difficult poses are detected better.
    	\item Width can largely be reduced.
    	\item With less than 9 layers performance deteriorates with every layer.
    \end{itemize}
    \end{frame}

    \begin{frame}{Next Steps}
    	\begin{itemize}
    		\item Analyzing what the deeper layers learn.
    		\item Investigate further methods to reduce model size/ speed up computations.
    	\end{itemize}
    \end{frame}
    

    \begin{frame}{Questions}
    \centering
\huge ?
    	\end{frame}
    
    
  \end{darkframes}

\end{document}
